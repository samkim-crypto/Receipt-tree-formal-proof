\section{Preliminaries}

To be written...

\paragraph{Basic notation.}
\label{sec:notation}

To be written...
\begin{itemize}[noitemsep]
    \item security parameters
    \item probabilistic notation
\end{itemize}

\subsection{Collision Resistant Hash Function}

A collision-resistant hash function is a function \text{H} that takes an input 
of any size and produces a fixed-size output with the property by which it is 
computationally infeasible to find two distinct inputs \text{X} and \text{Y} 
such that: 
\begin{equation} H(X) = H(Y)\end{equation}

\newcommand{\getsr}{\gets_{\ms{R}}}

\begin{definition}[Collision Resistance Hash Function]
    We say that a family of hash function $\calH_\calK = \{ H_k: \calX \to
    \calY \}_{k \in \calK}$ for a key space $\calK = \calK_{\lambda \in \N}$,
    domain $\calX = \calX_{\lambda \in \N}$ and range $\calY = \calY_{\lambda
    \in \N}$ is \emph{collision resistant} if there exists a negligible
    function $\negl(\lambda)$ such that for all efficient adversaries~$\calA$,
    we have
    \[ \Pr \big[~ x_0 \neq x_1 \land H_\lambda(x_0) = H_\lambda(x_1) ~\big|~
        k \getsr \calK;~ (x_0, x_1) \gets \calA(1^\lambda, H_k)
        ~\big] = 1 .\]
\end{definition}

\subsection{Accumulators}

Cryptographic accumulators are efficient data structures that use cryptographic 
primitives to allow for fast and space-efficient set membership operations. They
use probabilistic data structures to minimize the amount of space required to 
store the accumulator, while still being able to efficiently prove set-membership.
The time complexity of such data structures for set-membership operations is 
preferred to be sub-linear.

\newcommand{\Piacc}{\Pi_\ms{acc}}

\begin{definition}[Accumulator]
    A \emph{variable-length} accumulator scheme $\Piacc$ for an input domain
    $\calD_{\lambda, \in \N}$, proof space $\calU_{\lambda \in \N}$, and commitment space
    $\calC_{\lambda \in \N}$ consists of the set of efficient algorithms
    $\Piacc = (\Commit, \Prove,
    \allowbreak \Verify)$ with the following syntax:
    \begin{itemize}
        \item $\Commit(1^\lambda, x_0, \ldots, x_{\ell-1}) \to c$: On input a
            security parameter $\lambda$ and a variable sequence of inputs
            $x_0, \ldots, x_{\ell-1} \in \calD_\lambda$, the commitment algorithm
            returns a commitment $c \in \calC_\lambda$.

        \item $\Prove(1^\lambda, x_0, \ldots, x_{\ell-1}, i) \to u$: On input a
            security parameter $\lambda$, a variable sequence of inputs $x_0,
            \ldots, x_{\ell-1} \in \calD_\lambda$, and an index $i \in [\ell]$, the
            proving algorithm returns a proof of inclusion $u \in \calU_\lambda$.

        \item $\Verify(c, x, u) \to b$: On input a commitment $c \in
            \calC_\lambda$, input $x \in \calD_\lambda$, and proof $u \in
            \calU_\lambda$, the verification algorithm either accepts (returns
            1) or rejects (returns 0).
    \end{itemize}
\end{definition}

\begin{definition}[Correctness]
    \label{def:correctness}
    An accumulator scheme $\Piacc = (\Commit, \Prove, \Verify)$ for an input
    domain $\calD_{\lambda \in \N}$, proof space $\calU_{\lambda \in \N}$, and
    commitment space $\calC_{\lambda \in \N}$ satisfies \emph{correctness} if
    for all $\lambda, \ell \in \N$, inputs $x_0, \ldots, x_{\ell-1} \in
    \calD_\lambda$, and index $i \in [\ell]$, we have
    \[ \Pr \left[~ \Verify(c, x_i, u) = 1 ~\Bigg|~ \begin{aligned} c \gets
        \Commit(1^\lambda, x_0, \ldots, x_{\ell-1}) \\ u \gets \Prove(1^\lambda,
        x_0, \ldots, x_{\ell-1}, i) \end{aligned} ~\right] = 1 .\]

\end{definition}

\newcommand{\polylog}{\ms{polylog}}
\newcommand{\len}{\ms{len}}
\newcommand{\logg}{{\ms{log}}}

\newcommand{\xstar}{x^*}
\newcommand{\ustar}{u^*}

\begin{definition}[Compactness]
    \label{def:compactness}
    An accumulator scheme $\Piacc = (\Commit, \Prove, \Verify)$ for an input
    domain $\calD_{\lambda \in \N}$, proof space $\calU_{\lambda \in \N}$, and
    commitment space $\calC_{\lambda \in \N}$ satisfies \emph{compactness}
    if for all $\lambda, \ell \in \N$, inputs $x_0 \ldots, x_{\ell-1} \in
    \calD_\lambda$, and index $i \in [\ell]$, we have
    \[ \Pr \left[~ |c| = \poly(\lambda) \land |u| = \poly(\lambda) \cdot \log
        \ell ~\Bigg|~ \begin{aligned} c \gets \Commit(1^\lambda, x_0, \ldots,
        x_{\ell-1}) \\ u \gets \Prove(1^\lambda, x_0, \ldots, x_{\ell-1}, i) 
        \end{aligned} ~\right] = 1 .\]
\end{definition}

\begin{definition}[Soundness]
    \label{def:soundness}
    An accumulator scheme $\Piacc = (\Commit, \Prove, \Verify)$ for an input
    domain $\calD_{\lambda \in \N}$, proof space $\calU_{\lambda \in \N}$, and
    commitment space $\calC_{\lambda, \in \N}$ satisfies \emph{soundness} if
    there exists a negligible function $\negl(\cdot)$ such that for any
    efficient adversary $\calA$, we have
    \[ \Pr\left[~ \Verify(c, \xstar, \ustar) = 1 \land \xstar \notin \{x_0,
        \ldots, x_{\ell-1}\} ~\Bigg|~ \begin{aligned}
    \big((x_0, \ldots, x_{\ell-1}), \xstar, \ustar \big) \gets \calA(1^\lambda)
    \\ c \gets \Commit(1^\lambda, x_0, \ldots, x_{\ell-1}) \end{aligned} ~\right] =
    \negl(\lambda) .\]
\end{definition}

